导航需要地图。


我们使用的是二维地图(毕竟小哨兵还没学会飞行,在z轴上运动能力为0(其实还有roll和pitch),而且二维导航相关的导航,定位,建图方案都比较成熟)

\subsection{地图的获取}
首先,我们接收激光雷达扫描到的点云图,然后对其进行处理,压缩到二维,得到地图。

处理的第一步是进行点云的滤波。下面是一些常用的滤波方法:
\textbf{*注意:因为点云是通过物体表面反射光信息获取的,所以一个实体只有表面会有点云,内部是没有的。}
\begin{enumerate}
\item \textbf{Voxel滤波}:因为我们用的激光雷达比较牛逼,获得的近处点云比较稠密(保证远处的分辨率),这样直接计算会导致计算量过大,
所以我们可以将点云按照空间分辨率进行分割,又叫体素网格下采样。
\begin{tcode}
    pcl::VoxelGrid<pcl::PointXYZ> voxfilter;
    voxfilter.setInputCloud(cloud);
    // 设置滤波器的体素大小
    voxfilter.setLeafSize(0.1, 0.1, 0.1);
    voxfilter.filter(*cloud);
\end{tcode}
这样,我们就将cloud里每个0.1m*0.1m*0.1m的体素(三维空间中的一个小立方体单元)内的点替换为该体素的中心点。
将临近的点聚合到一个体素中来实现,可以减少点云数据的复杂度,同时保留整体的形状和结构。
\item \textbf{直通滤波}:直通滤波就是最简单的剔除不需要点云的滤波方法。比如小哨兵小小的,对它来说50cm以上的障碍物根本不影响他通过,那我们就可以把0.5m以上的点剔除掉。
\begin{tcode}
    pass_through_filter_z_.setFilterFieldName("z");
    pass_through_filter_z_.setFilterLimits(min_z, max_z);
    pass_through_filter_z_.setFilterLimitsNegative(false);
    pass_through_filter_z_.setInputCloud(input_cloud);
    pass_through_filter_z_.filter(*input_cloud);
\end{tcode}
这样,我们就将cloud里的z轴坐标在min\_z和max\_z之外的点剔除掉了。
同样的道理,还可以通过设置xy范围,把远处的,懒得考虑的点剔除掉。以及设置剔除一定半径内的点云,把哨兵本体的点云剔除掉。
\item \textbf{统计滤波}:在点云处理中,统计滤波是一种用于识别和移除离群点的方法,这些离群点可能是由于测量误差或噪声引起的。
该方法基于假设,即点云数据中的大部分点是符合高斯分布的,而离群点则是那些与这种分布显著不同的点。
\begin{tcode}
    pcl::StatisticalOutlierRemoval<pcl::PointXYZ> sor;
    sor.setInputCloud(cloud);
    // 设置近邻点的数量
    sor.setMeanK(50); 
    // 设置标准差倍数阈值
    sor.setStddevMulThresh(1.0); 
    // 设置为false表示移除离群点,设置为true表示保留离群点
    sor.setNegative(false); 
    sor.filter(*cloud);
\end{tcode}
通过这种方式,我们可以计算每个点到其K个最近邻点的平均距离,并根据这个平均距离与全局平均距离的比较来确定是否为离群点。
如果一个点的平均距离超过了全局平均距离加上标准差倍数的阈值,那么这个点就被认为是离群点并被移除。
这样可以有效地减少点云中的噪声和异常值,提高后续处理的准确性和效率。

\item \textbf{特殊方法滤波}:有时候一些特殊的物体需要滤掉,这时候就要自己编写特定的代码剔除。比如地板,是的,地板也是实体,也会产生点云,但他不是障碍物,所以需要过滤掉。
我们可以简单的通过直通滤波,把z轴坐标在0.1m以下的点剔除掉。也可以通过聚类(即将点云数据中的点根据它们的空间位置或属性特征分组),将最大的聚类剔除掉,这样就剔除了地板。
而我们现在用的是tc学长写的法向量剔除法,计算每个点的法向量,然后剔除法向量与z平面夹角小于一定角度的点(通俗来说:平的就是地板)。
\end{enumerate}


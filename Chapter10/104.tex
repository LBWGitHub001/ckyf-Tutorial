\subsection{全局规划器}

全局规划器是ROS中的核心模块之一,它负责规划机器人的全局路径,即从起始点到目标点的路径。
全局规划器的主要功能是生成一个全局路径,该路径是机器人从起始点到目标点的最短路径。
全局规划器的实现需要考虑机器人的移动方式、障碍物、地图等信息,以生成一个全局路径。

ROS中的全局规划器主要有两种实现方式:一种是基于RRT算法的全局规划器,另一种是基于PRM算法的全局规Planner。

\paragraph{基于RRT算法的全局规划器}

RRT算法是一种基于树形结构的路径规划算法,它通过随机生成点,然后根据两点之间的距离和角度来生成新的点,直到找到一个满足条件的点为止。
RRT算法的优点是简单易实现,且可以在任何地图上进行路径规划。但是,它的缺点是效率较低,
 especially when dealing with large maps or complex obstacles.

\subsection{局部规划器}
如果只有全局规划器会怎么样呢?现在,登录你舍友的LOL账号,开一把排位晋级赛,在小地图上点击敌方基地。
好的,一条全局导航路径生成出来了;好的,你的英雄沿着规划好的路径动起来了;好的,你被对面拿下一血并被队友问候。
这是为什么?因为敌方是动态的,他的位置不是事先知道的,事先知道的只有地图,而全局规划器只考虑了地图,未考虑动态障碍物。
局部规划器就是在你的视野里出现动态障碍物的时候,规划出一条道路绕开他,然后继续跟着全局路径走。

回到缺德地图,你这位车手就是局部规划器,缺德地图(全局规划器)只给大致的路线。
如何避让冲出来的小学生,掉头,超车,等红灯等都是车手(局部规划器)决定的。

常用的局部规划器有DWA,TEB等,我们使用的是